%INTRODUCAO - TCC-2 ALEX SOARES DUARTE
\chapter{Introdu\c{c}\~ao}
%Focar em uma coisa mais geral


%Falar sobre automacao industrial; sistemas de evento discreto; dificuldade de projeto
%Falar que tem um grande problema


%contextualizacao-1
Este trabalho propr\~oe uma ferramenta digital para a convers\~ao de sistemas modelados em Redes de Petri para c\'odigos implement\'aveis em controladores industriais.
Estamos cercados de sistemas a eventos discretos (SEDs). Um sistema a eventos discretos \'e definido como um sistema cuja evolu\c{c}\~ao din\^amica depende da ocorr\^encia de eventos, os quais produzem as mudan\c{c}as de estado e, de modo geral, ocorrem em instantes de tempo irregulares \cite{Montgomery2004}. 

Computadores, redes de comunica\c{c}\~oes, sistemas industriais automatizados, controle de tr\'afego a\'ereo s\~ao alguns exemplos de SEDs. Esses sistemas s\~ao modelados para que assim os usu\'arios possam, com o modelo, entender melhor o comportamento do sistema atrav\'es de an\'alises que os formalismos matem\'aticos proporcionam ao usu\'ario \cite{Wolfgang2013}.


V\'arios formalismos matem\'aticos podem ser considerados na modelagem de SEDs; Redes de Petri (RdP), Cadeias de Markov, teoria de linguagens e aut\^omatos (m\'aquinas de estados finitos) s\~ao alguns exemplos. Entretanto, n\~ao h\'a um formalismo universal que solucione todos os problemas referentes aos SEDs \cite{Montgomery2004}.
